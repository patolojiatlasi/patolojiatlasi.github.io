% Options for packages loaded elsewhere
\PassOptionsToPackage{unicode}{hyperref}
\PassOptionsToPackage{hyphens}{url}
\PassOptionsToPackage{dvipsnames,svgnames,x11names}{xcolor}
%
\documentclass[
  letterpaper,
  DIV=11,
  numbers=noendperiod]{scrreprt}
\usepackage{amsmath,amssymb}
\usepackage{lmodern}
\usepackage{iftex}
\ifPDFTeX
  \usepackage[T1]{fontenc}
  \usepackage[utf8]{inputenc}
  \usepackage{textcomp} % provide euro and other symbols
\else % if luatex or xetex
  \usepackage{unicode-math}
  \defaultfontfeatures{Scale=MatchLowercase}
  \defaultfontfeatures[\rmfamily]{Ligatures=TeX,Scale=1}
\fi
% Use upquote if available, for straight quotes in verbatim environments
\IfFileExists{upquote.sty}{\usepackage{upquote}}{}
\IfFileExists{microtype.sty}{% use microtype if available
  \usepackage[]{microtype}
  \UseMicrotypeSet[protrusion]{basicmath} % disable protrusion for tt fonts
}{}
\makeatletter
\@ifundefined{KOMAClassName}{% if non-KOMA class
  \IfFileExists{parskip.sty}{%
    \usepackage{parskip}
  }{% else
    \setlength{\parindent}{0pt}
    \setlength{\parskip}{6pt plus 2pt minus 1pt}}
}{% if KOMA class
  \KOMAoptions{parskip=half}}
\makeatother
\usepackage{xcolor}
\IfFileExists{xurl.sty}{\usepackage{xurl}}{} % add URL line breaks if available
\IfFileExists{bookmark.sty}{\usepackage{bookmark}}{\usepackage{hyperref}}
\hypersetup{
  pdftitle={Patoloji Atlası},
  pdfauthor={Serdar Balcı; Memorial Patoloji Hekim ve Teknikerleri},
  pdflang={tr},
  colorlinks=true,
  linkcolor={blue},
  filecolor={Maroon},
  citecolor={Blue},
  urlcolor={Blue},
  pdfcreator={LaTeX via pandoc}}
\urlstyle{same} % disable monospaced font for URLs
\usepackage{color}
\usepackage{fancyvrb}
\newcommand{\VerbBar}{|}
\newcommand{\VERB}{\Verb[commandchars=\\\{\}]}
\DefineVerbatimEnvironment{Highlighting}{Verbatim}{commandchars=\\\{\}}
% Add ',fontsize=\small' for more characters per line
\newenvironment{Shaded}{}{}
\newcommand{\AlertTok}[1]{\textcolor[rgb]{1.00,0.33,0.33}{\textbf{#1}}}
\newcommand{\AnnotationTok}[1]{\textcolor[rgb]{0.42,0.45,0.49}{#1}}
\newcommand{\AttributeTok}[1]{\textcolor[rgb]{0.84,0.23,0.29}{#1}}
\newcommand{\BaseNTok}[1]{\textcolor[rgb]{0.00,0.36,0.77}{#1}}
\newcommand{\BuiltInTok}[1]{\textcolor[rgb]{0.84,0.23,0.29}{#1}}
\newcommand{\CharTok}[1]{\textcolor[rgb]{0.01,0.18,0.38}{#1}}
\newcommand{\CommentTok}[1]{\textcolor[rgb]{0.42,0.45,0.49}{#1}}
\newcommand{\CommentVarTok}[1]{\textcolor[rgb]{0.42,0.45,0.49}{#1}}
\newcommand{\ConstantTok}[1]{\textcolor[rgb]{0.00,0.36,0.77}{#1}}
\newcommand{\ControlFlowTok}[1]{\textcolor[rgb]{0.84,0.23,0.29}{#1}}
\newcommand{\DataTypeTok}[1]{\textcolor[rgb]{0.84,0.23,0.29}{#1}}
\newcommand{\DecValTok}[1]{\textcolor[rgb]{0.00,0.36,0.77}{#1}}
\newcommand{\DocumentationTok}[1]{\textcolor[rgb]{0.42,0.45,0.49}{#1}}
\newcommand{\ErrorTok}[1]{\textcolor[rgb]{1.00,0.33,0.33}{\underline{#1}}}
\newcommand{\ExtensionTok}[1]{\textcolor[rgb]{0.84,0.23,0.29}{\textbf{#1}}}
\newcommand{\FloatTok}[1]{\textcolor[rgb]{0.00,0.36,0.77}{#1}}
\newcommand{\FunctionTok}[1]{\textcolor[rgb]{0.44,0.26,0.76}{#1}}
\newcommand{\ImportTok}[1]{\textcolor[rgb]{0.01,0.18,0.38}{#1}}
\newcommand{\InformationTok}[1]{\textcolor[rgb]{0.42,0.45,0.49}{#1}}
\newcommand{\KeywordTok}[1]{\textcolor[rgb]{0.84,0.23,0.29}{#1}}
\newcommand{\NormalTok}[1]{\textcolor[rgb]{0.14,0.16,0.18}{#1}}
\newcommand{\OperatorTok}[1]{\textcolor[rgb]{0.14,0.16,0.18}{#1}}
\newcommand{\OtherTok}[1]{\textcolor[rgb]{0.44,0.26,0.76}{#1}}
\newcommand{\PreprocessorTok}[1]{\textcolor[rgb]{0.84,0.23,0.29}{#1}}
\newcommand{\RegionMarkerTok}[1]{\textcolor[rgb]{0.42,0.45,0.49}{#1}}
\newcommand{\SpecialCharTok}[1]{\textcolor[rgb]{0.00,0.36,0.77}{#1}}
\newcommand{\SpecialStringTok}[1]{\textcolor[rgb]{0.01,0.18,0.38}{#1}}
\newcommand{\StringTok}[1]{\textcolor[rgb]{0.01,0.18,0.38}{#1}}
\newcommand{\VariableTok}[1]{\textcolor[rgb]{0.89,0.38,0.04}{#1}}
\newcommand{\VerbatimStringTok}[1]{\textcolor[rgb]{0.01,0.18,0.38}{#1}}
\newcommand{\WarningTok}[1]{\textcolor[rgb]{1.00,0.33,0.33}{#1}}
\usepackage{longtable,booktabs,array}
\usepackage{calc} % for calculating minipage widths
% Correct order of tables after \paragraph or \subparagraph
\usepackage{etoolbox}
\makeatletter
\patchcmd\longtable{\par}{\if@noskipsec\mbox{}\fi\par}{}{}
\makeatother
% Allow footnotes in longtable head/foot
\IfFileExists{footnotehyper.sty}{\usepackage{footnotehyper}}{\usepackage{footnote}}
\makesavenoteenv{longtable}
\usepackage{graphicx}
\makeatletter
\def\maxwidth{\ifdim\Gin@nat@width>\linewidth\linewidth\else\Gin@nat@width\fi}
\def\maxheight{\ifdim\Gin@nat@height>\textheight\textheight\else\Gin@nat@height\fi}
\makeatother
% Scale images if necessary, so that they will not overflow the page
% margins by default, and it is still possible to overwrite the defaults
% using explicit options in \includegraphics[width, height, ...]{}
\setkeys{Gin}{width=\maxwidth,height=\maxheight,keepaspectratio}
% Set default figure placement to htbp
\makeatletter
\def\fps@figure{htbp}
\makeatother
\setlength{\emergencystretch}{3em} % prevent overfull lines
\providecommand{\tightlist}{%
  \setlength{\itemsep}{0pt}\setlength{\parskip}{0pt}}
\setcounter{secnumdepth}{3}
\newlength{\cslhangindent}
\setlength{\cslhangindent}{1.5em}
\newlength{\csllabelwidth}
\setlength{\csllabelwidth}{3em}
\newlength{\cslentryspacingunit} % times entry-spacing
\setlength{\cslentryspacingunit}{\parskip}
\newenvironment{CSLReferences}[2] % #1 hanging-ident, #2 entry spacing
 {% don't indent paragraphs
  \setlength{\parindent}{0pt}
  % turn on hanging indent if param 1 is 1
  \ifodd #1
  \let\oldpar\par
  \def\par{\hangindent=\cslhangindent\oldpar}
  \fi
  % set entry spacing
  \setlength{\parskip}{#2\cslentryspacingunit}
 }%
 {}
\usepackage{calc}
\newcommand{\CSLBlock}[1]{#1\hfill\break}
\newcommand{\CSLLeftMargin}[1]{\parbox[t]{\csllabelwidth}{#1}}
\newcommand{\CSLRightInline}[1]{\parbox[t]{\linewidth - \csllabelwidth}{#1}\break}
\newcommand{\CSLIndent}[1]{\hspace{\cslhangindent}#1}
\ifLuaTeX
\usepackage[bidi=basic]{babel}
\else
\usepackage[bidi=default]{babel}
\fi
\babelprovide[main,import]{turkish}
% get rid of language-specific shorthands (see #6817):
\let\LanguageShortHands\languageshorthands
\def\languageshorthands#1{}
\KOMAoption{captions}{tableheading}
\makeatletter
\@ifpackageloaded{tcolorbox}{}{\usepackage[many]{tcolorbox}}
\@ifpackageloaded{fontawesome5}{}{\usepackage{fontawesome5}}
\definecolor{quarto-callout-color}{HTML}{909090}
\definecolor{quarto-callout-note-color}{HTML}{0758E5}
\definecolor{quarto-callout-important-color}{HTML}{CC1914}
\definecolor{quarto-callout-warning-color}{HTML}{EB9113}
\definecolor{quarto-callout-tip-color}{HTML}{00A047}
\definecolor{quarto-callout-caution-color}{HTML}{FC5300}
\definecolor{quarto-callout-color-frame}{HTML}{acacac}
\definecolor{quarto-callout-note-color-frame}{HTML}{4582ec}
\definecolor{quarto-callout-important-color-frame}{HTML}{d9534f}
\definecolor{quarto-callout-warning-color-frame}{HTML}{f0ad4e}
\definecolor{quarto-callout-tip-color-frame}{HTML}{02b875}
\definecolor{quarto-callout-caution-color-frame}{HTML}{fd7e14}
\makeatother
\makeatletter
\makeatother
\makeatletter
\@ifpackageloaded{caption}{}{\usepackage{caption}}
\AtBeginDocument{%
\renewcommand*\contentsname{Içindekiler}
\renewcommand*\listfigurename{Şekil Listesi}
\renewcommand*\listtablename{Tablo Listesi}
\renewcommand*\figurename{Figür}
\renewcommand*\tablename{Tablo}
}
\@ifpackageloaded{float}{}{\usepackage{float}}
\floatstyle{ruled}
\@ifundefined{c@chapter}{\newfloat{codelisting}{h}{lop}}{\newfloat{codelisting}{h}{lop}[chapter]}
\floatname{codelisting}{Listeleme}
\newcommand*\listoflistings{\listof{codelisting}{İlan Listesi}}
\makeatother
\makeatletter
\@ifpackageloaded{caption}{}{\usepackage{caption}}
\@ifpackageloaded{subcaption}{}{\usepackage{subcaption}}
\makeatother
\makeatletter
\@ifpackageloaded{tcolorbox}{}{\usepackage[many]{tcolorbox}}
\makeatother
\makeatletter
\@ifundefined{shadecolor}{\definecolor{shadecolor}{rgb}{.97, .97, .97}}
\makeatother
\makeatletter
\makeatother
\ifLuaTeX
  \usepackage{selnolig}  % disable illegal ligatures
\fi

\title{Patoloji Atlası}
\usepackage{etoolbox}
\makeatletter
\providecommand{\subtitle}[1]{% add subtitle to \maketitle
  \apptocmd{\@title}{\par {\large #1 \par}}{}{}
}
\makeatother
\subtitle{Patoloji Atlası: Tıp Fakültesi ve Sağlık Bilimleri Öğrencileri
İçin Patoloji Laboratuvar Notları: Görerek Öğrenin}
\author{Serdar Balcı \and Memorial Patoloji Hekim ve Teknikerleri}
\date{2022-03-23T22:17:52+03:00}

\begin{document}
\maketitle
\begin{abstract}
Patoloji Atlası: Tıp Fakültesi ve Sağlık Bilimleri Öğrencileri İçin
Patoloji Laboratuvar Notları. Görerek Öğrenin. Patoloji Atlası Memorial
Patoloji arşivinden derlenen vakalardan oluşmaktadır.
\href{https://www.patolojiatlasi.com/katki.html}{Katkı yapmak ve kendi
vakalarınız ekletmek için lütfen iletişime geçin}.
\end{abstract}

\ifdefined\Shaded\renewenvironment{Shaded}{\begin{tcolorbox}[sharp corners, frame hidden, enhanced, boxrule=0pt, interior hidden, borderline west={3pt}{0pt}{shadecolor}]}{\end{tcolorbox}}\fi

\renewcommand*\contentsname{İçindekiler}
{
\hypersetup{linkcolor=}
\setcounter{tocdepth}{2}
\tableofcontents
}
Patoloji Atlası

\href{/EN/}{For English click here}

\href{/GBD/}{GBDAtlas}

Giriş

Patoloji Atlası Memorial Patoloji arşivinden derlenen vakalardan
oluşmaktadır.\\
\href{https://www.patolojiatlasi.com/katki.html}{Katkı yapmak ve kendi
vakalarınız ekletmek için lütfen iletişime geçin}.

Son güncelleme zamanı: 2022-03-23 22:32:52

Sosyal medyadan derlenen görüntülerden oluşan
\href{https://www.patolojinotlari.com/}{patoloji notları için
tıklayınız}.

Yazarlar ve Katkıda Bulunanlar

\hypertarget{derleyen}{%
\chapter*{Derleyen:}\label{derleyen}}
\addcontentsline{toc}{chapter}{Derleyen:}

\begin{itemize}
\tightlist
\item
  \href{https://www.serdarbalci.com}{Serdar Balcı}
\end{itemize}

\hypertarget{katkux131da-bulunanlar}{%
\chapter*{Katkıda Bulunanlar:}\label{katkux131da-bulunanlar}}
\addcontentsline{toc}{chapter}{Katkıda Bulunanlar:}

\textbf{\href{https://MemorialPath.github.io}{Memorial Patoloji}
Hekimleri}

\begin{itemize}
\item
  \href{https://www.memorial.com.tr/en/doctors/ilknur-turkmen-1975}{Ilknur
  Turkmen}
\item
  {[}Gülen Bülbül Doğusoy{]}\\
  \strut \\
  \strut \\
  \strut \\
\item
  {[}Serdar Balcı{]}\\
\item
  {[}Yıldırım Karslıoğlu{]}
\item
  {[}Mehtat Uz Ünlü{]}
\item
  {[}Murat Oktay{]}\\
  \strut \\
  \strut \\
  \strut \\
  \strut \\
\end{itemize}

\textbf{\href{https://MemorialPath.github.io}{Memorial Patoloji}
Teknikerleri}

\begin{itemize}
\item
  {[}Emrah Uça{]}
\item
  {[}Elif Sevin Şanioğlu{]}
\end{itemize}

Genel Patoloji

Hücre İçi Birikimler

\hypertarget{kolesterol-polibi}{%
\chapter{Kolesterol Polibi}\label{kolesterol-polibi}}

\begin{itemize}
\item
  \url{https://pathologyatlas.github.io/cholesterolpolyp/HE.html}
\item
  See Microscopy with viewer:
\end{itemize}

\hypertarget{glikojen-depo-hastalux131ux11fux131}{%
\chapter{Glikojen Depo
Hastalığı}\label{glikojen-depo-hastalux131ux11fux131}}

Karaciğer İğnde Biyopsisinde glikojen depo hastalığı

\textbf{Hematoksilen Eozin}

\url{https://pathologyatlas.github.io/glycogenstorage/HE.html}

Mikroskopik görüntüleri inceleyin:

\textbf{PAS}

\url{https://pathologyatlas.github.io/glycogenstorage/PAS.html}

Mikroskopik görüntüleri inceleyin:

\textbf{PASD}

\url{https://pathologyatlas.github.io/glycogenstorage/PASD.html}

Mikroskopik görüntüleri inceleyin:

\hypertarget{antrakoz-antrakotik-pigment}{%
\chapter{Antrakoz, Antrakotik
Pigment}\label{antrakoz-antrakotik-pigment}}

Torakal bölge lenf nodunda antrakotik pigment

\url{https://pathologyatlas.github.io/anthracosis/HE.html}

See Microscopy with viewer:

\hypertarget{melanosis-coli}{%
\chapter{Melanosis Coli}\label{melanosis-coli}}

\textbf{Melanozis Koli}

\url{https://pathologyatlas.github.io/melanosiscoli/HE.html}

Mikroskopik görüntüleri inceleyin:

\textbf{Melanozis Koli PAS}

\url{https://pathologyatlas.github.io/melanosiscoli/PAS.html}

Mikroskopik görüntüleri inceleyin:

Okronozis

\textbf{Okronozis}

\url{https://pathologyatlas.github.io/ochronosis/HE.html}

Mikroskopik görüntüleri inceleyin:

Hemodinamik Bozukluklar

\hypertarget{iskemi-ve-nekroz}{%
\chapter{İskemi ve Nekroz}\label{iskemi-ve-nekroz}}

\hypertarget{yaux11f-nekrozu-ve-sabunlaux15fma}{%
\section{Yağ nekrozu ve
Sabunlaşma}\label{yaux11f-nekrozu-ve-sabunlaux15fma}}

Yağ dokuda yağ nekrozu ve sabunlaşma

\url{https://pathologyatlas.github.io/fat-necrosis/HE.html}

Mikroskopik görüntüleri inceleyin:

Amiloidoz (Amiloid Birikimi)

\hypertarget{kristal-viyole}{%
\chapter{Kristal Viyole}\label{kristal-viyole}}

Damar duvarlarında amiloid birikimi

\url{https://pathologyatlas.github.io/amyloid/crystalviolet.html}

Mikroskopik görüntüleri inceleyin:

\hypertarget{congo-red}{%
\chapter{Congo Red}\label{congo-red}}

Congo Red stain for amyloidosis

\url{https://pathologyatlas.github.io/congored/congored.html}

See Microscopy with viewer:

\hypertarget{congo-red-birefringence}{%
\chapter{Congo Red Birefringence}\label{congo-red-birefringence}}

\hypertarget{tamir-mekanizmalarux131}{%
\chapter{Tamir Mekanizmaları}\label{tamir-mekanizmalarux131}}

\hypertarget{fibrozis}{%
\section{Fibrozis}\label{fibrozis}}

Kolesistit spesmeninde gelişmekte olan genç fibrozis

\url{https://pathologyatlas.github.io/fibrosis/HE.html}

Mikroskopik görüntüleri inceleyin:

\hypertarget{keloid---skar}{%
\section{Keloid - Skar}\label{keloid---skar}}

Keloid Skar oluşumu

\url{https://pathologyatlas.github.io/keloid-scar/HE.html}

Mikroskopik görüntüleri inceleyin:

İnflamasyon

Kronik İnflamasyon

\hypertarget{hidronefroz-ve-kronik-pyelonefrit}{%
\chapter{Hidronefroz ve Kronik
Pyelonefrit}\label{hidronefroz-ve-kronik-pyelonefrit}}

Hidronefroz ve Kronik Pyelonefrit

\url{https://pathologyatlas.github.io/chronicpyelonephritis/HE1.html}

Mikroskopik görüntüleri inceleyin:

\url{https://pathologyatlas.github.io/chronicpyelonephritis/HE2.html}

Mikroskopik görüntüleri inceleyin:

Nekrotizan Granülamatöz İnflamasyon

\textbf{Karaciğer dokusunda nekrotizan granülamatöz inflamasyon}

\url{https://pathologyatlas.github.io/necrotisinggranuloma/HE.html}

Mikroskopik görüntüleri inceleyin:

İnfeksiyöz Hastalıkların Patolojisi

Viruslar

\hypertarget{herpes-simplex-virus-hsv}{%
\chapter{Herpes Simplex Virus (HSV)}\label{herpes-simplex-virus-hsv}}

\hypertarget{herpes-esophagatis}{%
\section{Herpes Esophagatis}\label{herpes-esophagatis}}

\url{https://pathologyatlas.github.io/HSV/herpesesophagitis/viewer_z0.html}

Mikroskopik görüntüleri inceleyin:

\hypertarget{molluscum-contagiosum}{%
\chapter{Molluscum contagiosum}\label{molluscum-contagiosum}}

Molluscum contagiosum

\textbf{Molluscum contagiosum}

\url{https://pathologyatlas.github.io/molluscum-contagiosum/HE.html}

Mikroskopik görüntüleri inceleyin:

Bakteriler

\textbf{Mide'de Helicobacter pylori (H. pylori) HE}

\url{https://pathologyatlas.github.io/helicobacterpylori/HE.html}

Mikroskopik görüntüleri inceleyin:

\textbf{Mide'de Helicobacter pylori (H. pylori) Warthin Starry
Histokimyası}

\url{https://pathologyatlas.github.io/helicobacterpylori/warthinstarry.html}

Mikroskopik görüntüleri inceleyin:

\textbf{Mide'de Helicobacter pylori (H. pylori) Giemsa Histokimyası}

\url{https://pathologyatlas.github.io/helicobacterpylori/giemsa.html}

Mikroskopik görüntüleri inceleyin:

\hypertarget{mantarlar}{%
\chapter{Mantarlar}\label{mantarlar}}

\hypertarget{candida-albicans-in-cervicovaginal-smear}{%
\section{\texorpdfstring{Candida \emph{albicans} in cervicovaginal
smear}{Candida albicans in cervicovaginal smear}}\label{candida-albicans-in-cervicovaginal-smear}}

\url{https://pathologyatlas.github.io/candidaalbicans/cervicovaginalsmear/viewer_z0.html}

Mikroskopik görüntüleri inceleyin:

Enterobius vermicularis

\textbf{Enterobius vermicularis}

\url{https://pathologyatlas.github.io/enterobius-vermicularis/HE.html}

Mikroskopik görüntüleri inceleyin:

Temel Tümör Patolojisi

Benign Tümörler

\hypertarget{adenomlar}{%
\chapter{Adenomlar}\label{adenomlar}}

\hypertarget{tuxfcbuxfcler-adenom}{%
\section{Tübüler Adenom}\label{tuxfcbuxfcler-adenom}}

\hypertarget{sesil-polip-flat-duxfcz-tuxfcbuxfcler-adenom}{%
\subsection{Sesil Polip, Flat (Düz) Tübüler
Adenom}\label{sesil-polip-flat-duxfcz-tuxfcbuxfcler-adenom}}

\url{https://pathologyatlas.github.io/tubularadenoma-flat/HE.html}

Mikroskopik görüntüleri inceleyin:

\url{https://pathologyatlas.github.io/tubularadenoma-flat/HE2.html}

Mikroskopik görüntüleri inceleyin:

\hypertarget{saplux131-polip}{%
\subsection{Saplı Polip}\label{saplux131-polip}}

Macroscopy

\href{https://pathologyatlas.github.io/tubularadenoma/tubular-adenoma-with-stalk-macroscopy.jpg}{tubular
adenoma with a stalk macroscopy}

Microscopy

\href{https://pathologyatlas.github.io/tubularadenoma/tubular-adenoma-with-stalk.jpeg}{tubular
adenoma with a stalk}

Mikroskopik görüntüleri inceleyin:

\url{https://pathologyatlas.github.io/tubularadenoma/tubular-adenoma-with-stalk/viewer_z0.html}

Hamartom

\hypertarget{hamartomatuxf6z-polip}{%
\chapter{Hamartomatöz Polip}\label{hamartomatuxf6z-polip}}

\begin{itemize}
\item
  \url{https://pathologyatlas.github.io/hamartomatouspolyp/HE.html}
\item
  Mikroskopik görüntüleri inceleyin:
\end{itemize}

\hypertarget{schwann-cell-hamartoma-in-a-colon-polyp}{%
\chapter{Schwann Cell Hamartoma in a Colon
Polyp}\label{schwann-cell-hamartoma-in-a-colon-polyp}}

\begin{itemize}
\item
  \url{https://pathologyatlas.github.io/schwanncellhamartoma/HE.html}
\item
  Mikroskopik görüntüleri inceleyin:
\end{itemize}

Heterotopi

\hypertarget{intrapancreatic-spleen-heterotopia}{%
\chapter{Intrapancreatic Spleen,
Heterotopia}\label{intrapancreatic-spleen-heterotopia}}

\begin{itemize}
\item
  \url{https://pathologyatlas.github.io/intrapancreaticspleen/HE.html}
\item
  Mikroskopik görüntüleri inceleyin:
\end{itemize}

Metastaz

\hypertarget{karaciux11ferde-sarkom-metastazux131}{%
\chapter{Karaciğerde Sarkom
Metastazı}\label{karaciux11ferde-sarkom-metastazux131}}

\begin{itemize}
\item
  \url{https://pathologyatlas.github.io/metastaticsarcoma/HE.html}
\item
  Mikroskopik görüntüleri inceleyin:
\end{itemize}

\hypertarget{v2}{%
\chapter{V2}\label{v2}}

\href{./metastaticsarcoma/HE.html}{metastaticsarcoma/HE.html}

Mikroskopik görüntüleri inceleyin:

Tümörlerdeki Prognostik Morfolojik Özellikler

Adenokarsinomda Ekstramural Venöz İnvazyon

\textbf{Adenokarsinomda Ekstramural Venöz İnvazyon}

\url{https://pathologyatlas.github.io/extramuralvenousinvasion/HE.html}

Mikroskopik görüntüleri inceleyin:

Gastrointestinal Sistem Patolojisi

Gastrointestinal sistem Patolojisi

Kolon

\hypertarget{benign-tuxfcmuxf6rler-1}{%
\chapter{Benign Tümörler}\label{benign-tuxfcmuxf6rler-1}}

\hypertarget{tuxfcbuxfcler-adenom-1}{%
\section{Tübüler Adenom}\label{tuxfcbuxfcler-adenom-1}}

\hypertarget{sesil-polip-flat-duxfcz-tuxfcbuxfcler-adenom-1}{%
\subsection{Sesil Polip, Flat (Düz) Tübüler
Adenom}\label{sesil-polip-flat-duxfcz-tuxfcbuxfcler-adenom-1}}

\url{https://pathologyatlas.github.io/tubularadenoma-flat/HE.html}

Mikroskopik görüntüleri inceleyin:

\url{https://pathologyatlas.github.io/tubularadenoma-flat/HE2.html}

Mikroskopik görüntüleri inceleyin:

\hypertarget{saplux131-polip-1}{%
\subsection{Saplı Polip}\label{saplux131-polip-1}}

Macroscopy

\href{https://pathologyatlas.github.io/tubularadenoma/tubular-adenoma-with-stalk-macroscopy.jpg}{tubular
adenoma with a stalk macroscopy}

\begin{figure}

{\centering \includegraphics{https://pathologyatlas.github.io/tubularadenoma/tubular-adenoma-with-stalk-macroscopy.jpg}

}

\caption{saplı polip}

\end{figure}

Microscopy

\href{https://pathologyatlas.github.io/tubularadenoma/tubular-adenoma-with-stalk.jpeg}{tubular
adenoma with a stalk}

Mikroskopik görüntüleri inceleyin:

\url{https://pathologyatlas.github.io/tubularadenoma/tubular-adenoma-with-stalk/viewer_z0.html}

\hypertarget{hiperplastik-polip}{%
\section{Hiperplastik Polip}\label{hiperplastik-polip}}

Mikroskopi

\url{https://pathologyatlas.github.io/hyperplasticpolyp/case1.html}

Mikroskopik görüntüleri inceleyin:

\hypertarget{hamartomatuxf6z-polip-1}{%
\section{Hamartomatöz Polip}\label{hamartomatuxf6z-polip-1}}

\url{https://pathologyatlas.github.io/hamartomatouspolyp/HE.html}

Mikroskopik görüntüleri inceleyin:

\hypertarget{schwann-cell-hamartoma-in-a-colon-polyp-1}{%
\chapter{Schwann Cell Hamartoma in a Colon
Polyp}\label{schwann-cell-hamartoma-in-a-colon-polyp-1}}

\url{https://pathologyatlas.github.io/schwanncellhamartoma/HE.html}

Mikroskopik görüntüleri inceleyin:

Karaciğer Patolojisi

Karaciğer Tümörleri

\hypertarget{hepatoseluxfcler-karsinom}{%
\section{Hepatoselüler Karsinom}\label{hepatoseluxfcler-karsinom}}

\url{https://pathologyatlas.github.io/hepatocellularcarcinoma/HCC/viewer_z0.html}

Mikroskopik görüntüleri inceleyin:

Jinekolojik Patoloji

Jinekopatoloji

\hypertarget{over}{%
\chapter{Over}\label{over}}

\hypertarget{ovary-serous-borderline-tumor-micropapillary-variant}{%
\section{Ovary, Serous borderline tumor, micropapillary
variant}\label{ovary-serous-borderline-tumor-micropapillary-variant}}

\url{https://pathologyatlas.github.io/ovarianserousmicropapillary/HE.html}

Mikroskopik görüntüleri inceleyin:

Genital Sistem Patolojisi

Üriner Sistem Patolojisi

Uriner Sistem Patolojisi

Böbrek Tümörleri

\hypertarget{onkositom}{%
\chapter{Onkositom}\label{onkositom}}

\url{https://pathologyatlas.github.io/kidneyoncocytoma/HE.html}

Mikroskopik görüntüleri inceleyin:

Meme Patolojisi

Breast Pathology

See extended list of Breast Pathology cases here
\url{https://pathologyatlas.github.io/breast/}

Baş Boyun Patolojisi

\hypertarget{kulak}{%
\chapter{Kulak}\label{kulak}}

\hypertarget{kolesteatom}{%
\section{Kolesteatom}\label{kolesteatom}}

\url{https://pathologyatlas.github.io/cholesteatoma/cholesteatoma.html}

Mikroskopik görüntüleri inceleyin:

Santral Sinir Sistem Patolojisi

Yüksek Dereceli Gliom Yayma Preparat

\textbf{Yüksek Dereceli Gliom Yayma Preparat}

\url{https://pathologyatlas.github.io/high-grade-glioma-squash/HE.html}

Mikroskopik görüntüleri inceleyin:

Beyin İnvazyonu Gösteren Meningiom

\textbf{Beyin İnvazyonu Gösteren Meningiom}

\url{https://pathologyatlas.github.io/brain-invasive-meningioma/HE.html}

Mikroskopik görüntüleri inceleyin:

Meningiom Kemik İnfiltrasyonu

\textbf{Meningiom Kemik İnfiltrasyonu}

\url{https://pathologyatlas.github.io/meningioma-bone-infiltration/HE.html}

Mikroskopik görüntüleri inceleyin:

Kemik Patolojisi

Kemik Tümörleri

\hypertarget{benign-kemik-tuxfcmuxf6rleri}{%
\chapter{Benign Kemik Tümörleri}\label{benign-kemik-tuxfcmuxf6rleri}}

\hypertarget{ekzostoz-osteokondrom}{%
\section{Ekzostoz (Osteokondrom)}\label{ekzostoz-osteokondrom}}

\url{https://pathologyatlas.github.io/exostosis/oc.html}

Mikroskopik görüntüleri inceleyin:

\url{https://pathologyatlas.github.io/exostosis/oc001.html}

Mikroskopik görüntüleri inceleyin:

\url{https://pathologyatlas.github.io/exostosis/oc002.html}

Mikroskopik görüntüleri inceleyin:

\hypertarget{section}{%
\section{}\label{section}}

Yumuşak Doku Patolojisi

Miksoid Liposarkom

\textbf{Miksoid Liposarkom}

\url{https://pathologyatlas.github.io/myxoidliposarcoma/HE.html}

Mikroskopik görüntüleri inceleyin:

Pediatrik ve Perinatal Patoloji

Plasenta

\hypertarget{koryoamnionit}{%
\chapter{Koryoamnionit}\label{koryoamnionit}}

\url{https://pathologyatlas.github.io/chorioamnionitis/HE.html}

Mikroskopik görüntüleri inceleyin:

Referanslar

\hypertarget{refs}{}
\begin{CSLReferences}{0}{0}
\end{CSLReferences}

Ekler

Yönetim ve Geliştirme

\href{https://pathologyatlas.github.io/}{pathologyatlas.github.io}

\href{https://lab.patolojinotlari.com}{lab.patolojinotlari.com}

\href{https://patolojinotlari.com}{patolojinotlari.com}

\href{https://parapathology.com}{parapathology.com}

\href{https://twitter.com/patolojinotlari}{twitter}

\href{https://www.linkedin.com/company/patoloji-notlari}{linkedin}

\url{https://leanpub.com/patolojiatlasi/}

\href{https://pathologyatlas.github.io/development.md}{development \&
WIP}

\begin{itemize}
\item
  \url{https://github.com/pathologyatlas}
\item
  \url{https://github.com/pathologyatlas/TODO}
\item
  \url{https://github.com/pathologyatlas/pathologyatlas.github.io}
\item
  \url{https://github.com/pathologyatlas/template}
\item
  \url{https://github.com/pathologyatlas/make-html-WSI}
\end{itemize}

contact: info@patolojinotlari.com

contact: bilgi@patolojiatlasi.com

\begin{itemize}
\item
  https://github.com/marketplace/actions/quarto-render
\item
  https://github.com/quarto-dev/quarto-actions
\end{itemize}

\begin{center}\rule{0.5\linewidth}{0.5pt}\end{center}

Examples for future uploads

\hypertarget{pancreas-ductal-adenocarcinoma}{%
\chapter{Pancreas Ductal
Adenocarcinoma}\label{pancreas-ductal-adenocarcinoma}}

\url{https://pathologyatlas.github.io/pancreaticadenocarcinoma/}

\hypertarget{case-1}{%
\section{Case 1}\label{case-1}}

\href{https://pathologyatlas.github.io/pancreaticadenocarcinoma/case1-histopathology/viewer_z0.html}{histopathology}

\hypertarget{neoplazinin-klinikopatolojik-uxf6zellikleri-ve-epidemiyoloji}{%
\chapter{Neoplazinin Klinikopatolojik Özellikleri ve
Epidemiyoloji}\label{neoplazinin-klinikopatolojik-uxf6zellikleri-ve-epidemiyoloji}}

\url{https://pathologyatlas.github.io/lecture1/Neoplazinin-Klinikopatolojik-Ozellikleri-ve-Epidemiyoloji.html}

\begin{center}\rule{0.5\linewidth}{0.5pt}\end{center}

adding submodule

https://github.blog/2016-02-01-working-with-submodules/

\begin{Shaded}
\begin{Highlighting}[]
\FunctionTok{git}\NormalTok{ submodule add https://github.com/pathologyatlas/pancreaticadenocarcinoma pancreaticadenocarcinoma}
\end{Highlighting}
\end{Shaded}

removing submodule

\begin{Shaded}
\begin{Highlighting}[]
\FunctionTok{git}\NormalTok{ submodule deinit }\AttributeTok{{-}f} \AttributeTok{{-}{-}all}
\FunctionTok{rm} \AttributeTok{{-}rf}\NormalTok{ .git/modules/}
\FunctionTok{git}\NormalTok{ rm }\AttributeTok{{-}f}\NormalTok{ pancreaticadenocarcinoma}
\end{Highlighting}
\end{Shaded}

making WSI

see:
\href{https://github.com/pathologyatlas/make-html-WSI\#convert-svs-to-dzi-and-publish-as-or-embed-in-a-web-page}{Convert
.svs to .dzi and publish as or embed in a web page}

using templates for new repositories

see: \href{https://github.com/pathologyatlas/template}{Using templates
for new repositories}

\begin{center}\rule{0.5\linewidth}{0.5pt}\end{center}

TODO

\hypertarget{yuxf6nlendirme}{%
\chapter{Yönlendirme}\label{yuxf6nlendirme}}

\begin{itemize}
\tightlist
\item
  development ve wsi to html repolarının adreslerini güncelle ve
  detaylandır
\item
  https://quarto.org/docs/websites/website-navigation.html\#pages-404
\item
  Yönlendirme örneği: https://github.com/patolojiatlasi/GBD
\end{itemize}

\hypertarget{gbdatlas}{%
\chapter{GBDAtlas}\label{gbdatlas}}

\begin{itemize}
\tightlist
\item
  GBDAtlas ilk grubu ekle
\end{itemize}

\hypertarget{language}{%
\chapter{Language}\label{language}}

\begin{itemize}
\tightlist
\item
  https://statnmap.com/2017-04-04-format-text-conditionally-with-rmarkdown-chunks-complete-iframed-article/
\item
  https://statnmap.com/2017-10-06-translation-rmarkdown-documents-using-data-frame/
\item
  https://statnmap.com/2017-03-11-rmarkdown-conditional-chunks-to-create-multilingual-pdf-and-html-with-images/
\item
  https://bookdown.org/yihui/rmarkdown-cookbook/eng-asis.html
\item
  https://quarto.org/docs/authoring/create-citeable-articles.html
\item
  https://github.com/maelle/multilingual-book
\item
  https://github.com/patolojiatlasi/multilingual-book
\end{itemize}

\hypertarget{github-actions}{%
\chapter{GitHub Actions}\label{github-actions}}

\begin{itemize}
\tightlist
\item
  https://github.com/quarto-dev/quarto-actions
\item
  https://github.com/pommevilla/quarto-render
\item
  https://github.com/pommevilla/friendly-dollop
\item
  https://pommevilla.github.io/friendly-dollop/
\end{itemize}

\hypertarget{tweeting-with-update}{%
\chapter{tweeting with update}\label{tweeting-with-update}}

\begin{itemize}
\tightlist
\item
  https://github.com/and-computers/HowToTweetEveryCommit
\item
  https://www.daveabrock.com/2020/04/19/posting-to-twitter-from-gh-actions/
\item
  https://gitweet.io/
\item
  https://joshuaiz.com/words/using-github-actions-to-post-to-twitter-on-commit
\end{itemize}

\begin{center}\rule{0.5\linewidth}{0.5pt}\end{center}

\begin{tcolorbox}[standard jigsaw,title=\textcolor{quarto-callout-note-color}{\faInfo}\hspace{0.5em}Not, rightrule=.15mm, opacitybacktitle=0.6, leftrule=.75mm, colback=white, left=2mm, colframe=quarto-callout-note-color-frame, toprule=.15mm, bottomtitle=1mm, coltitle=black, titlerule=0mm, colbacktitle=quarto-callout-note-color!10!white, opacityback=0, toptitle=1mm, bottomrule=.15mm, arc=.35mm]
Note that there are five types of callouts, including: \texttt{note},
\texttt{tip}, \texttt{warning}, \texttt{caution}, and
\texttt{important}.
\end{tcolorbox}

\begin{tcolorbox}[standard jigsaw,title=\textcolor{quarto-callout-tip-color}{\faLightbulb}\hspace{0.5em}Tip With Caption, rightrule=.15mm, opacitybacktitle=0.6, leftrule=.75mm, colback=white, left=2mm, colframe=quarto-callout-tip-color-frame, toprule=.15mm, bottomtitle=1mm, coltitle=black, titlerule=0mm, colbacktitle=quarto-callout-tip-color!10!white, opacityback=0, toptitle=1mm, bottomrule=.15mm, arc=.35mm]
This is an example of a callout with a caption.
\end{tcolorbox}

\begin{tcolorbox}[standard jigsaw,title=\textcolor{quarto-callout-caution-color}{\faFire}\hspace{0.5em}Expand To Learn About Collapse, rightrule=.15mm, opacitybacktitle=0.6, leftrule=.75mm, colback=white, left=2mm, colframe=quarto-callout-caution-color-frame, toprule=.15mm, bottomtitle=1mm, coltitle=black, titlerule=0mm, colbacktitle=quarto-callout-caution-color!10!white, opacityback=0, toptitle=1mm, bottomrule=.15mm, arc=.35mm]
This is an example of a `folded' caution callout that can be expanded by
the user. You can use \texttt{collapse="true"} to collapse it by default
or \texttt{collapse="false"} to make a collapsible callout that is
expanded by default.
\end{tcolorbox}

Katkıda Bulunmak İçin

H\&E ya da özel boyalı preparatlarınızı bize elden ya da kargo ile
ulaştırırsanız, onları tarayıp atlasa ekleyebiliriz.

Halihazırda taranmış preparatlarınızı bulut üzerinden
\href{mailto:bilgi@patolojiatlasi.com}{eposta} ile paylaşırsanız, onları
da ekleyebiliriz.

Preparatın olduğu sayfaya eklenmek üzere

\begin{verbatim}
Adınızı,

Kurumunuzu

Yönlendirilmesini istediğiniz iletişim linkini (web ya da e-posta)

Eklemek istediğiniz ek klinik bilgi, resim ya da notları
\end{verbatim}

\href{mailto:bilgi@patolojiatlasi.com}{\nolinkurl{bilgi@patolojiatlasi.com}}
adresine iletiniz.

Adres: Doç.Dr.Serdar Balcı Memorial Patoloji Laboratuvarı, Ortadoğu
Plaza Kat:14, Kaptanpaşa Mah. Piyalepaşa Bulvarı, Okmeydanı Cd No:73,
34384 Şişli/İstanbul

\end{document}
