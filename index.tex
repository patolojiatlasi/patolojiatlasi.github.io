% Options for packages loaded elsewhere
\PassOptionsToPackage{unicode}{hyperref}
\PassOptionsToPackage{hyphens}{url}
\PassOptionsToPackage{dvipsnames,svgnames,x11names}{xcolor}
%
\documentclass[
  letterpaper,
  DIV=11,
  numbers=noendperiod]{scrreprt}

\usepackage{amsmath,amssymb}
\usepackage{lmodern}
\usepackage{iftex}
\ifPDFTeX
  \usepackage[T1]{fontenc}
  \usepackage[utf8]{inputenc}
  \usepackage{textcomp} % provide euro and other symbols
\else % if luatex or xetex
  \usepackage{unicode-math}
  \defaultfontfeatures{Scale=MatchLowercase}
  \defaultfontfeatures[\rmfamily]{Ligatures=TeX,Scale=1}
\fi
% Use upquote if available, for straight quotes in verbatim environments
\IfFileExists{upquote.sty}{\usepackage{upquote}}{}
\IfFileExists{microtype.sty}{% use microtype if available
  \usepackage[]{microtype}
  \UseMicrotypeSet[protrusion]{basicmath} % disable protrusion for tt fonts
}{}
\makeatletter
\@ifundefined{KOMAClassName}{% if non-KOMA class
  \IfFileExists{parskip.sty}{%
    \usepackage{parskip}
  }{% else
    \setlength{\parindent}{0pt}
    \setlength{\parskip}{6pt plus 2pt minus 1pt}}
}{% if KOMA class
  \KOMAoptions{parskip=half}}
\makeatother
\usepackage{xcolor}
\setlength{\emergencystretch}{3em} % prevent overfull lines
\setcounter{secnumdepth}{5}
% Make \paragraph and \subparagraph free-standing
\ifx\paragraph\undefined\else
  \let\oldparagraph\paragraph
  \renewcommand{\paragraph}[1]{\oldparagraph{#1}\mbox{}}
\fi
\ifx\subparagraph\undefined\else
  \let\oldsubparagraph\subparagraph
  \renewcommand{\subparagraph}[1]{\oldsubparagraph{#1}\mbox{}}
\fi

\usepackage{color}
\usepackage{fancyvrb}
\newcommand{\VerbBar}{|}
\newcommand{\VERB}{\Verb[commandchars=\\\{\}]}
\DefineVerbatimEnvironment{Highlighting}{Verbatim}{commandchars=\\\{\}}
% Add ',fontsize=\small' for more characters per line
\newenvironment{Shaded}{}{}
\newcommand{\AlertTok}[1]{\textcolor[rgb]{1.00,0.00,0.00}{\textbf{#1}}}
\newcommand{\AnnotationTok}[1]{\textcolor[rgb]{0.38,0.63,0.69}{\textbf{\textit{#1}}}}
\newcommand{\AttributeTok}[1]{\textcolor[rgb]{0.49,0.56,0.16}{#1}}
\newcommand{\BaseNTok}[1]{\textcolor[rgb]{0.25,0.63,0.44}{#1}}
\newcommand{\BuiltInTok}[1]{#1}
\newcommand{\CharTok}[1]{\textcolor[rgb]{0.25,0.44,0.63}{#1}}
\newcommand{\CommentTok}[1]{\textcolor[rgb]{0.38,0.63,0.69}{\textit{#1}}}
\newcommand{\CommentVarTok}[1]{\textcolor[rgb]{0.38,0.63,0.69}{\textbf{\textit{#1}}}}
\newcommand{\ConstantTok}[1]{\textcolor[rgb]{0.53,0.00,0.00}{#1}}
\newcommand{\ControlFlowTok}[1]{\textcolor[rgb]{0.00,0.44,0.13}{\textbf{#1}}}
\newcommand{\DataTypeTok}[1]{\textcolor[rgb]{0.56,0.13,0.00}{#1}}
\newcommand{\DecValTok}[1]{\textcolor[rgb]{0.25,0.63,0.44}{#1}}
\newcommand{\DocumentationTok}[1]{\textcolor[rgb]{0.73,0.13,0.13}{\textit{#1}}}
\newcommand{\ErrorTok}[1]{\textcolor[rgb]{1.00,0.00,0.00}{\textbf{#1}}}
\newcommand{\ExtensionTok}[1]{#1}
\newcommand{\FloatTok}[1]{\textcolor[rgb]{0.25,0.63,0.44}{#1}}
\newcommand{\FunctionTok}[1]{\textcolor[rgb]{0.02,0.16,0.49}{#1}}
\newcommand{\ImportTok}[1]{#1}
\newcommand{\InformationTok}[1]{\textcolor[rgb]{0.38,0.63,0.69}{\textbf{\textit{#1}}}}
\newcommand{\KeywordTok}[1]{\textcolor[rgb]{0.00,0.44,0.13}{\textbf{#1}}}
\newcommand{\NormalTok}[1]{#1}
\newcommand{\OperatorTok}[1]{\textcolor[rgb]{0.40,0.40,0.40}{#1}}
\newcommand{\OtherTok}[1]{\textcolor[rgb]{0.00,0.44,0.13}{#1}}
\newcommand{\PreprocessorTok}[1]{\textcolor[rgb]{0.74,0.48,0.00}{#1}}
\newcommand{\RegionMarkerTok}[1]{#1}
\newcommand{\SpecialCharTok}[1]{\textcolor[rgb]{0.25,0.44,0.63}{#1}}
\newcommand{\SpecialStringTok}[1]{\textcolor[rgb]{0.73,0.40,0.53}{#1}}
\newcommand{\StringTok}[1]{\textcolor[rgb]{0.25,0.44,0.63}{#1}}
\newcommand{\VariableTok}[1]{\textcolor[rgb]{0.10,0.09,0.49}{#1}}
\newcommand{\VerbatimStringTok}[1]{\textcolor[rgb]{0.25,0.44,0.63}{#1}}
\newcommand{\WarningTok}[1]{\textcolor[rgb]{0.38,0.63,0.69}{\textbf{\textit{#1}}}}

\providecommand{\tightlist}{%
  \setlength{\itemsep}{0pt}\setlength{\parskip}{0pt}}\usepackage{longtable,booktabs,array}
\usepackage{calc} % for calculating minipage widths
% Correct order of tables after \paragraph or \subparagraph
\usepackage{etoolbox}
\makeatletter
\patchcmd\longtable{\par}{\if@noskipsec\mbox{}\fi\par}{}{}
\makeatother
% Allow footnotes in longtable head/foot
\IfFileExists{footnotehyper.sty}{\usepackage{footnotehyper}}{\usepackage{footnote}}
\makesavenoteenv{longtable}
\usepackage{graphicx}
\makeatletter
\def\maxwidth{\ifdim\Gin@nat@width>\linewidth\linewidth\else\Gin@nat@width\fi}
\def\maxheight{\ifdim\Gin@nat@height>\textheight\textheight\else\Gin@nat@height\fi}
\makeatother
% Scale images if necessary, so that they will not overflow the page
% margins by default, and it is still possible to overwrite the defaults
% using explicit options in \includegraphics[width, height, ...]{}
\setkeys{Gin}{width=\maxwidth,height=\maxheight,keepaspectratio}
% Set default figure placement to htbp
\makeatletter
\def\fps@figure{htbp}
\makeatother
\newlength{\cslhangindent}
\setlength{\cslhangindent}{1.5em}
\newlength{\csllabelwidth}
\setlength{\csllabelwidth}{3em}
\newlength{\cslentryspacingunit} % times entry-spacing
\setlength{\cslentryspacingunit}{\parskip}
\newenvironment{CSLReferences}[2] % #1 hanging-ident, #2 entry spacing
 {% don't indent paragraphs
  \setlength{\parindent}{0pt}
  % turn on hanging indent if param 1 is 1
  \ifodd #1
  \let\oldpar\par
  \def\par{\hangindent=\cslhangindent\oldpar}
  \fi
  % set entry spacing
  \setlength{\parskip}{#2\cslentryspacingunit}
 }%
 {}
\usepackage{calc}
\newcommand{\CSLBlock}[1]{#1\hfill\break}
\newcommand{\CSLLeftMargin}[1]{\parbox[t]{\csllabelwidth}{#1}}
\newcommand{\CSLRightInline}[1]{\parbox[t]{\linewidth - \csllabelwidth}{#1}\break}
\newcommand{\CSLIndent}[1]{\hspace{\cslhangindent}#1}

\KOMAoption{captions}{tableheading}
\makeatletter
\@ifpackageloaded{tcolorbox}{}{\usepackage[many]{tcolorbox}}
\@ifpackageloaded{fontawesome5}{}{\usepackage{fontawesome5}}
\definecolor{quarto-callout-color}{HTML}{909090}
\definecolor{quarto-callout-note-color}{HTML}{0758E5}
\definecolor{quarto-callout-important-color}{HTML}{CC1914}
\definecolor{quarto-callout-warning-color}{HTML}{EB9113}
\definecolor{quarto-callout-tip-color}{HTML}{00A047}
\definecolor{quarto-callout-caution-color}{HTML}{FC5300}
\definecolor{quarto-callout-color-frame}{HTML}{acacac}
\definecolor{quarto-callout-note-color-frame}{HTML}{4582ec}
\definecolor{quarto-callout-important-color-frame}{HTML}{d9534f}
\definecolor{quarto-callout-warning-color-frame}{HTML}{f0ad4e}
\definecolor{quarto-callout-tip-color-frame}{HTML}{02b875}
\definecolor{quarto-callout-caution-color-frame}{HTML}{fd7e14}
\makeatother
\makeatletter
\makeatother
\makeatletter
\@ifpackageloaded{bookmark}{}{\usepackage{bookmark}}
\makeatother
\makeatletter
\@ifpackageloaded{caption}{}{\usepackage{caption}}
\AtBeginDocument{%
\ifdefined\contentsname
  \renewcommand*\contentsname{Içindekiler}
\else
  \newcommand\contentsname{Içindekiler}
\fi
\ifdefined\listfigurename
  \renewcommand*\listfigurename{Şekil Listesi}
\else
  \newcommand\listfigurename{Şekil Listesi}
\fi
\ifdefined\listtablename
  \renewcommand*\listtablename{Tablo Listesi}
\else
  \newcommand\listtablename{Tablo Listesi}
\fi
\ifdefined\figurename
  \renewcommand*\figurename{Figür}
\else
  \newcommand\figurename{Figür}
\fi
\ifdefined\tablename
  \renewcommand*\tablename{Tablo}
\else
  \newcommand\tablename{Tablo}
\fi
}
\@ifpackageloaded{float}{}{\usepackage{float}}
\floatstyle{ruled}
\@ifundefined{c@chapter}{\newfloat{codelisting}{h}{lop}}{\newfloat{codelisting}{h}{lop}[chapter]}
\floatname{codelisting}{Listeleme}
\newcommand*\listoflistings{\listof{codelisting}{İlan Listesi}}
\makeatother
\makeatletter
\@ifpackageloaded{caption}{}{\usepackage{caption}}
\@ifpackageloaded{subcaption}{}{\usepackage{subcaption}}
\makeatother
\makeatletter
\makeatother
\ifLuaTeX
\usepackage[bidi=basic]{babel}
\else
\usepackage[bidi=default]{babel}
\fi
\babelprovide[main,import]{turkish}
% get rid of language-specific shorthands (see #6817):
\let\LanguageShortHands\languageshorthands
\def\languageshorthands#1{}
\ifLuaTeX
  \usepackage{selnolig}  % disable illegal ligatures
\fi
\IfFileExists{bookmark.sty}{\usepackage{bookmark}}{\usepackage{hyperref}}
\IfFileExists{xurl.sty}{\usepackage{xurl}}{} % add URL line breaks if available
\urlstyle{same} % disable monospaced font for URLs
\hypersetup{
  pdftitle={Patoloji Atlası},
  pdfauthor={Serdar Balcı; Memorial Patoloji Hekim ve Teknikerleri},
  pdflang={tr},
  colorlinks=true,
  linkcolor={blue},
  filecolor={Maroon},
  citecolor={Blue},
  urlcolor={Blue},
  pdfcreator={LaTeX via pandoc}}

\title{Patoloji Atlası}
\usepackage{etoolbox}
\makeatletter
\providecommand{\subtitle}[1]{% add subtitle to \maketitle
  \apptocmd{\@title}{\par {\large #1 \par}}{}{}
}
\makeatother
\subtitle{Patoloji Atlası: Tıp Fakültesi ve Sağlık Bilimleri Öğrencileri
İçin Patoloji Laboratuvar Notları: Görerek Öğrenin}
\author{Serdar Balcı \and Memorial Patoloji Hekim ve Teknikerleri}
\date{2022-08-15T08:26:18+03:00}

\begin{document}
\maketitle
\begin{abstract}
Patoloji Atlası: Tıp Fakültesi ve Sağlık Bilimleri Öğrencileri İçin
Patoloji Laboratuvar Notları. Görerek Öğrenin. Patoloji Atlası Memorial
Patoloji arşivinden derlenen vakalardan oluşmaktadır. Katkı yapmak ve
kendi vakalarınız ekletmek için lütfen
\href{https://www.patolojiatlasi.com/katki.html}{iletişime geçin}.
\end{abstract}
\renewcommand*\contentsname{Içindekiler}
{
\hypersetup{linkcolor=}
\setcounter{tocdepth}{2}
\tableofcontents
}
\bookmarksetup{startatroot}

\hypertarget{patoloji-atlasi}{%
\chapter*{Patoloji Atlası}\label{patoloji-atlasi}}
\addcontentsline{toc}{chapter}{Patoloji Atlası}

For English \href{/EN/}{click here}.

\href{/GBD/}{GBDAtlas}

Sosyal medyadan derlenen görüntülerden oluşan patoloji notları için
\href{https://www.patolojinotlari.com/}{tıklayınız}.

\textbf{Atıf için}

\begin{longtable}[]{@{}
  >{\raggedright\arraybackslash}p{(\columnwidth - 0\tabcolsep) * \real{1.0000}}@{}}
\toprule()
\endhead
\href{https://zenodo.org/badge/latestdoi/452585667}{\includegraphics{https://zenodo.org/badge/452585667.svg}} \\
\href{https://osf.io/6w5k8/}{Open Science Framework DOI:
10.17605/OSF.IO/6W5K8} \\
\href{https://github.com/patolojiatlasi/patolojiatlasi.github.io/issues}{\includegraphics{https://img.shields.io/github/issues/patolojiatlasi/patolojiatlasi.github.io}} \\
\href{https://github.com/patolojiatlasi/patolojiatlasi.github.io/blob/main/LICENSE}{\includegraphics{https://img.shields.io/github/license/patolojiatlasi/patolojiatlasi.github.io}} \\
 \\
 \\
\bottomrule()
\end{longtable}

\bookmarksetup{startatroot}

\hypertarget{giris}{%
\chapter*{Giriş}\label{giris}}
\addcontentsline{toc}{chapter}{Giriş}

Patoloji Atlası Memorial Patoloji arşivinden derlenen vakalardan
oluşmaktadır.\\

\href{https://www.patolojiatlasi.com/katki.html}{Katkı yapmak ve kendi
vakalarınız ekletmek için lütfen iletişime geçin}.

Son güncelleme zamanı: 2022-08-15 09:30:03

Sosyal medyadan derlenen görüntülerden oluşan
\href{https://www.patolojinotlari.com/}{patoloji notları için
tıklayınız}.

\bookmarksetup{startatroot}

\hypertarget{yazarlar-ve-katkux131da-bulunanlar}{%
\chapter*{Yazarlar ve Katkıda
Bulunanlar}\label{yazarlar-ve-katkux131da-bulunanlar}}
\addcontentsline{toc}{chapter}{Yazarlar ve Katkıda Bulunanlar}

\hypertarget{derleyen}{%
\section*{Derleyen:}\label{derleyen}}
\addcontentsline{toc}{section}{Derleyen:}

\begin{itemize}
\tightlist
\item
  \href{https://www.serdarbalci.com}{Serdar Balcı}
\end{itemize}

\hypertarget{katkux131da-bulunanlar}{%
\section*{Katkıda Bulunanlar:}\label{katkux131da-bulunanlar}}
\addcontentsline{toc}{section}{Katkıda Bulunanlar:}

\href{https://patoloji.memorial.com.tr/}{\textbf{Memorial Patoloji}}
\textbf{Hekimleri}

\begin{itemize}
\item
  \href{https://www.memorial.com.tr/en/doctors/ilknur-turkmen-1975}{Ilknur
  Turkmen}
\item
  \href{https://www.memorial.com.tr/doktorlar/gulen-bulbul-dogusoy}{Gülen
  Bülbül Doğusoy}
\item
  \href{https://www.memorial.com.tr/doktorlar/fatma-aktepe}{Fatma
  Aktepe}
\item
  \href{https://www.memorial.com.tr/doktorlar/turkan-atasever-rezanko}{Türkan
  Atasever Rezanko}
\item
  \href{https://www.memorial.com.tr/doktorlar/pembe-gul-gunes}{Pembe Gül
  Güneş}
\item
  \href{https://www.memorial.com.tr/doktorlar/semsi-yildiz}{Şemsi
  Yıldız}
\item
  \href{https://www.memorial.com.tr/doktorlar/serdar-balci}{Serdar
  Balcı}
\item
  \href{https://www.memorial.com.tr/doktorlar/sezen-kocarslan}{Sezen
  Koçarslan}
\item
  \href{https://www.memorial.com.tr/doktorlar/yildirim-karslioglu}{Yıldırım
  Karslıoğlu}
\item
  \href{https://www.memorial.com.tr/doktorlar/mehtat-uz-unlu}{Mehtat Uz
  Ünlü}
\item
  \href{https://www.memorial.com.tr/doktorlar/murat-oktay}{Murat Oktay}
\item
  \href{https://www.memorial.com.tr/doktorlar/deniz-baycelebi}{Deniz
  Bayçelebi}
\item
  \href{https://www.memorial.com.tr/doktorlar/emre-karakok}{Emre
  Karakök}
\item
  \href{https://www.memorial.com.tr/doktorlar/zuhal-kus-silav}{Zuhal Kuş
  Sılav}
\item
  \href{https://www.memorial.com.tr/doktorlar/fatma-gulgun-sade-kocak}{Fatma
  Gülgün Sade Koçak}
\item
  \href{https://www.memorial.com.tr/doktorlar/zeynep-pehlivanoglu}{Zeynep
  Pehlivanoğlu}
\end{itemize}

\href{https://patoloji.memorial.com.tr/}{\textbf{Memorial Patoloji}}
\textbf{Teknikerleri}

\begin{itemize}
\item
  {[}Emrah Uça{]}
\item
  \href{https://www.linkedin.com/in/\%C5\%9Fevin-elif-\%C5\%9Fanio\%C4\%9Flu-99449a1b0/}{Şevin
  Elif Şanioğlu}
\item
  \href{https://www.linkedin.com/in/baset-s\%C4\%B1\%C4\%9F\%C4\%B1rc\%C4\%B1-aa2406141/}{Baset
  Sığırcı}
\item
  \href{https://www.linkedin.com/in/rabia-\%C3\%B6zt\%C3\%BCrk-4989b3151/}{Rabia
  Öztürk}
\end{itemize}

\part{---}

\part{Genel Patoloji}

\hypertarget{huxfccre-iuxe7i-birikimler}{%
\chapter{Hücre İçi Birikimler}\label{huxfccre-iuxe7i-birikimler}}

\hypertarget{kolesterol-polibi}{%
\section{Kolesterol Polibi}\label{kolesterol-polibi}}

\begin{itemize}
\item
  \url{https://pathologyatlas.github.io/cholesterolpolyp/HE.html}
\item
  See Microscopy with viewer:
\end{itemize}

\hypertarget{glikojen-depo-hastalux131ux11fux131}{%
\section{Glikojen Depo
Hastalığı}\label{glikojen-depo-hastalux131ux11fux131}}

Karaciğer İğnde Biyopsisinde glikojen depo hastalığı

\textbf{Hematoksilen Eozin}

\url{https://pathologyatlas.github.io/glycogenstorage/HE.html}

Mikroskopik görüntüleri inceleyin:

\textbf{PAS}

\url{https://pathologyatlas.github.io/glycogenstorage/PAS.html}

Mikroskopik görüntüleri inceleyin:

\textbf{PASD}

\url{https://pathologyatlas.github.io/glycogenstorage/PASD.html}

Mikroskopik görüntüleri inceleyin:

\hypertarget{antrakoz-antrakotik-pigment}{%
\section{Antrakoz, Antrakotik
Pigment}\label{antrakoz-antrakotik-pigment}}

Torakal bölge lenf nodunda antrakotik pigment

\url{https://pathologyatlas.github.io/anthracosis/HE.html}

See Microscopy with viewer:

\hypertarget{melanosis-coli}{%
\section{Melanosis Coli}\label{melanosis-coli}}

\textbf{Melanozis Koli}

\url{https://pathologyatlas.github.io/melanosiscoli/HE.html}

Mikroskopik görüntüleri inceleyin:

\textbf{Melanozis Koli PAS}

\url{https://pathologyatlas.github.io/melanosiscoli/PAS.html}

Mikroskopik görüntüleri inceleyin:

\hypertarget{huxfccre-dux131ux15fux131-birikimler}{%
\chapter{Hücre Dışı
Birikimler}\label{huxfccre-dux131ux15fux131-birikimler}}

\hypertarget{okronozis}{%
\section{Okronozis}\label{okronozis}}

\textbf{Okronozis}

\url{https://pathologyatlas.github.io/ochronosis/HE.html}

Mikroskopik görüntüleri inceleyin:

\hypertarget{huxfccre-hasarux131}{%
\chapter{Hücre Hasarı}\label{huxfccre-hasarux131}}

\hypertarget{reaktif-atipi-uxfclsere-kolon-polibi}{%
\section{Reaktif Atipi, ülsere kolon
polibi}\label{reaktif-atipi-uxfclsere-kolon-polibi}}

\textbf{Reaktif Atipi, ülsere kolon polibi}

\url{https://pathologyatlas.github.io/reactive-atypia/HE.html}

Mikroskopik görüntüleri inceleyin:

\hypertarget{hemodinamik-bozukluklar}{%
\chapter{Hemodinamik Bozukluklar}\label{hemodinamik-bozukluklar}}

\hypertarget{iskemi-ve-nekroz}{%
\section{İskemi ve Nekroz}\label{iskemi-ve-nekroz}}

\hypertarget{yaux11f-nekrozu-ve-sabunlaux15fma}{%
\subsection{Yağ nekrozu ve
Sabunlaşma}\label{yaux11f-nekrozu-ve-sabunlaux15fma}}

Yağ dokuda yağ nekrozu ve sabunlaşma

\url{https://pathologyatlas.github.io/fat-necrosis/HE.html}

Mikroskopik görüntüleri inceleyin:

\hypertarget{amiloidoz-amiloid-birikimi}{%
\chapter{Amiloidoz (Amiloid
Birikimi)}\label{amiloidoz-amiloid-birikimi}}

\hypertarget{kristal-viyole}{%
\section{Kristal Viyole}\label{kristal-viyole}}

Damar duvarlarında amiloid birikimi

\url{https://pathologyatlas.github.io/amyloid/crystalviolet.html}

Mikroskopik görüntüleri inceleyin:

\hypertarget{congo-red}{%
\section{Congo Red}\label{congo-red}}

Congo Red stain for amyloidosis

\url{https://pathologyatlas.github.io/congored/congored.html}

See Microscopy with viewer:

\hypertarget{congo-red-birefringence}{%
\section{Congo Red Birefringence}\label{congo-red-birefringence}}

\hypertarget{tamir-mekanizmalarux131}{%
\chapter{Tamir Mekanizmaları}\label{tamir-mekanizmalarux131}}

\hypertarget{fibrozis}{%
\section{Fibrozis}\label{fibrozis}}

Kolesistit spesmeninde gelişmekte olan genç fibrozis

\url{https://pathologyatlas.github.io/fibrosis/HE.html}

Mikroskopik görüntüleri inceleyin:

\hypertarget{keloid---skar}{%
\section{Keloid - Skar}\label{keloid---skar}}

Keloid Skar oluşumu

\url{https://pathologyatlas.github.io/keloid-scar/HE.html}

Mikroskopik görüntüleri inceleyin:

\part{İnflamasyon}

\hypertarget{kronik-inflamasyon}{%
\chapter{Kronik İnflamasyon}\label{kronik-inflamasyon}}

\hypertarget{hidronefroz-ve-kronik-pyelonefrit}{%
\section{Hidronefroz ve Kronik
Pyelonefrit}\label{hidronefroz-ve-kronik-pyelonefrit}}

Hidronefroz ve Kronik Pyelonefrit

\url{https://pathologyatlas.github.io/chronicpyelonephritis/HE1.html}

Mikroskopik görüntüleri inceleyin:

\url{https://pathologyatlas.github.io/chronicpyelonephritis/HE2.html}

Mikroskopik görüntüleri inceleyin:

\hypertarget{granuxfclamatuxf6z-inflamasyon}{%
\chapter{Granülamatöz
İnflamasyon}\label{granuxfclamatuxf6z-inflamasyon}}

\hypertarget{nekrotizan-granuxfclamatuxf6z-inflamasyon}{%
\section{Nekrotizan Granülamatöz
İnflamasyon}\label{nekrotizan-granuxfclamatuxf6z-inflamasyon}}

\textbf{Karaciğer dokusunda nekrotizan granülamatöz inflamasyon}

\url{https://pathologyatlas.github.io/necrotisinggranuloma/HE.html}

Mikroskopik görüntüleri inceleyin:

\part{İnfeksiyöz Hastalıkların Patolojisi}

\hypertarget{viruslar}{%
\chapter{Viruslar}\label{viruslar}}

\hypertarget{herpes-simplex-virus-hsv}{%
\section{Herpes Simplex Virus (HSV)}\label{herpes-simplex-virus-hsv}}

\hypertarget{herpes-esophagatis}{%
\subsection{Herpes Esophagatis}\label{herpes-esophagatis}}

\url{https://pathologyatlas.github.io/HSV/herpesesophagitis/viewer_z0.html}

Mikroskopik görüntüleri inceleyin:

\hypertarget{molluscum-contagiosum}{%
\section{Molluscum contagiosum}\label{molluscum-contagiosum}}

\textbf{Molluscum contagiosum}

\url{https://pathologyatlas.github.io/molluscum-contagiosum/HE.html}

Mikroskopik görüntüleri inceleyin:

\hypertarget{bakteriler}{%
\chapter{Bakteriler}\label{bakteriler}}

\hypertarget{helicobacter-pylori}{%
\section{Helicobacter pylori}\label{helicobacter-pylori}}

\textbf{Mide'de Helicobacter pylori (H. pylori) HE}

\url{https://pathologyatlas.github.io/helicobacterpylori/HE.html}

Mikroskopik görüntüleri inceleyin:

\textbf{Mide'de Helicobacter pylori (H. pylori) Warthin Starry
Histokimyası}

\url{https://pathologyatlas.github.io/helicobacterpylori/warthinstarry.html}

Mikroskopik görüntüleri inceleyin:

\textbf{Mide'de Helicobacter pylori (H. pylori) Giemsa Histokimyası}

\url{https://pathologyatlas.github.io/helicobacterpylori/giemsa.html}

Mikroskopik görüntüleri inceleyin:

\hypertarget{mantarlar}{%
\chapter{Mantarlar}\label{mantarlar}}

\hypertarget{candida-albicans-in-cervicovaginal-smear}{%
\section{\texorpdfstring{Candida \emph{albicans} in cervicovaginal
smear}{Candida albicans in cervicovaginal smear}}\label{candida-albicans-in-cervicovaginal-smear}}

\begin{Shaded}
\begin{Highlighting}[]
\FunctionTok{print}\NormalTok{(}\StringTok{"repeating content"}\NormalTok{)}
\end{Highlighting}
\end{Shaded}

\begin{verbatim}
[1] "repeating content"
\end{verbatim}

\url{https://pathologyatlas.github.io/candidaalbicans/cervicovaginalsmear/viewer_z0.html}

Mikroskopik görüntüleri inceleyin:

\hypertarget{beyin-mukormikozis}{%
\section{Beyin mukormikozis}\label{beyin-mukormikozis}}

\textbf{Beyin mukormikozis HE}

\url{https://pathologyatlas.github.io/brain-mucormycosis/HE.html}

Mikroskopik görüntüleri inceleyin:

\textbf{Beyin mukormikozis GMS}

\url{https://pathologyatlas.github.io/brain-mucormycosis/HE.html}

Mikroskopik görüntüleri inceleyin:

\hypertarget{parazitler}{%
\chapter{Parazitler}\label{parazitler}}

\hypertarget{enterobius-vermicularis}{%
\section{Enterobius vermicularis}\label{enterobius-vermicularis}}

\textbf{Enterobius vermicularis}

\url{https://pathologyatlas.github.io/enterobius-vermicularis/HE.html}

Mikroskopik görüntüleri inceleyin:

\part{Temel Tümör Patolojisi}

\hypertarget{benign-tuxfcmuxf6rler}{%
\chapter{Benign Tümörler}\label{benign-tuxfcmuxf6rler}}

\hypertarget{adenomlar}{%
\section{Adenomlar}\label{adenomlar}}

\hypertarget{tuxfcbuxfcler-adenom}{%
\subsection{Tübüler Adenom}\label{tuxfcbuxfcler-adenom}}

\hypertarget{sesil-polip-flat-duxfcz-tuxfcbuxfcler-adenom}{%
\subsubsection{Sesil Polip, Flat (Düz) Tübüler
Adenom}\label{sesil-polip-flat-duxfcz-tuxfcbuxfcler-adenom}}

\url{https://pathologyatlas.github.io/tubularadenoma-flat/HE.html}

Mikroskopik görüntüleri inceleyin:

\url{https://pathologyatlas.github.io/tubularadenoma-flat/HE2.html}

Mikroskopik görüntüleri inceleyin:

\hypertarget{saplux131-polip}{%
\subsubsection{Saplı Polip}\label{saplux131-polip}}

Macroscopy

\href{https://pathologyatlas.github.io/tubularadenoma/tubular-adenoma-with-stalk-macroscopy.jpg}{tubular
adenoma with a stalk macroscopy}

Microscopy

\href{https://pathologyatlas.github.io/tubularadenoma/tubular-adenoma-with-stalk.jpeg}{tubular
adenoma with a stalk}

Mikroskopik görüntüleri inceleyin:

\url{https://pathologyatlas.github.io/tubularadenoma/tubular-adenoma-with-stalk/viewer_z0.html}

\hypertarget{hamartom}{%
\chapter{Hamartom}\label{hamartom}}

\hypertarget{hamartomatuxf6z-polip}{%
\section{Hamartomatöz Polip}\label{hamartomatuxf6z-polip}}

\begin{itemize}
\item
  \url{https://pathologyatlas.github.io/hamartomatouspolyp/HE.html}
\item
  Mikroskopik görüntüleri inceleyin:
\end{itemize}

\hypertarget{schwann-cell-hamartoma-in-a-colon-polyp}{%
\section{Schwann Cell Hamartoma in a Colon
Polyp}\label{schwann-cell-hamartoma-in-a-colon-polyp}}

\begin{itemize}
\item
  \url{https://pathologyatlas.github.io/schwanncellhamartoma/HE.html}
\item
  Mikroskopik görüntüleri inceleyin:
\end{itemize}

\hypertarget{heterotopi-ektopi}{%
\chapter{Heterotopi Ektopi}\label{heterotopi-ektopi}}

\hypertarget{intrapancreatic-spleen-heterotopia}{%
\section{Intrapancreatic Spleen,
Heterotopia}\label{intrapancreatic-spleen-heterotopia}}

\begin{itemize}
\item
  \url{https://pathologyatlas.github.io/intrapancreaticspleen/HE.html}
\item
  Mikroskopik görüntüleri inceleyin:
\end{itemize}

\hypertarget{paratubal-adneksiyal-buxf6lgede-ektopik-adrenal-dokusu}{%
\section{Paratubal adneksiyal bölgede ektopik adrenal
dokusu}\label{paratubal-adneksiyal-buxf6lgede-ektopik-adrenal-dokusu}}

\textbf{Paratubal adneksiyal bölgede ektopik adrenal dokusu}

\url{https://pathologyatlas.github.io/ectopic-adrenal/HE.html}

Mikroskopik görüntüleri inceleyin:

knitr::include\_url

\begin{Shaded}
\begin{Highlighting}[]
\NormalTok{knitr}\SpecialCharTok{::}\FunctionTok{include\_url}\NormalTok{(}\AttributeTok{url =} \StringTok{"https://pathologyatlas.github.io/ectopic{-}adrenal/HE.html"}\NormalTok{)}
\end{Highlighting}
\end{Shaded}

\begin{figure}[H]

{\centering 

\href{https://pathologyatlas.github.io/ectopic-adrenal/HE.html}{\includegraphics{./heterotopi_files/figure-pdf/unnamed-chunk-5-1.pdf}}

}

\end{figure}

\hypertarget{metastaz}{%
\chapter{Metastaz}\label{metastaz}}

\begin{Shaded}
\begin{Highlighting}[]
\FunctionTok{print}\NormalTok{(}\StringTok{"repeating content"}\NormalTok{)}
\end{Highlighting}
\end{Shaded}

\begin{verbatim}
[1] "repeating content"
\end{verbatim}

\hypertarget{karaciux11ferde-sarkom-metastazux131}{%
\section{Karaciğerde Sarkom
Metastazı}\label{karaciux11ferde-sarkom-metastazux131}}

\begin{itemize}
\item
  \url{https://pathologyatlas.github.io/metastaticsarcoma/HE.html}
\item
  Mikroskopik görüntüleri inceleyin:
\end{itemize}

\hypertarget{sinsi-bir-lenf-nodu-metastazux131}{%
\section{Sinsi bir lenf nodu
metastazı}\label{sinsi-bir-lenf-nodu-metastazux131}}

\textbf{Sinsi bir lenf nodu metastazı HE}

\url{https://pathologyatlas.github.io/insidious-lymph-node-metastasis/HE.html}

Mikroskopik görüntüleri inceleyin:

\textbf{Sinsi bir lenf nodu metastazı OSKAR panCK}

\url{https://pathologyatlas.github.io/insidious-lymph-node-metastasis/OSKARCK.html}

Mikroskopik görüntüleri inceleyin:

\part{Tümörlerdeki Prognostik Morfolojik Özellikler}

\hypertarget{venuxf6z-invazyon}{%
\chapter{Venöz invazyon}\label{venuxf6z-invazyon}}

\textbf{Venöz invazyon}

\url{https://pathologyatlas.github.io/venous-invasion/HE.html}

Mikroskopik görüntüleri inceleyin:

\hypertarget{adenokarsinomda-ekstramural-venuxf6z-invazyon}{%
\chapter{Adenokarsinomda Ekstramural Venöz
İnvazyon}\label{adenokarsinomda-ekstramural-venuxf6z-invazyon}}

\textbf{Adenokarsinomda Ekstramural Venöz İnvazyon}

\url{https://pathologyatlas.github.io/extramuralvenousinvasion/HE.html}

Mikroskopik görüntüleri inceleyin:

\part{Gastrointestinal Sistem Patolojisi}

\hypertarget{gastrointestinal-sistem-patolojisi-1}{%
\chapter{Gastrointestinal sistem
Patolojisi}\label{gastrointestinal-sistem-patolojisi-1}}

\hypertarget{uxf6zefagusta-granuxfcler-huxfccreli-tuxfcmuxf6r}{%
\chapter{Özefagusta Granüler Hücreli
Tümör}\label{uxf6zefagusta-granuxfcler-huxfccreli-tuxfcmuxf6r}}

\textbf{Özefagusta Granüler Hücreli Tümör}

\url{https://pathologyatlas.github.io/granular-cell-tumor/HE.html}

Mikroskopik görüntüleri inceleyin:

\hypertarget{mide-patolojisi}{%
\chapter{Mide Patolojisi}\label{mide-patolojisi}}

\hypertarget{gastritis-cystica-profunda}{%
\section{Gastritis Cystica Profunda}\label{gastritis-cystica-profunda}}

\textbf{Gastritis Cystica Profunda}

\url{https://pathologyatlas.github.io/gastritis-cystica-profunda/HE.html}

Mikroskopik görüntüleri inceleyin:

\hypertarget{kolon-patolojisi}{%
\chapter{Kolon patolojisi}\label{kolon-patolojisi}}

Serdar Balcı\\
last-modified

\hfill\break

\textbf{İskemik Kolit}

\url{https://pathologyatlas.github.io/ischemic-colitis/HE.html}

Mikroskopik görüntüleri inceleyin:

\hypertarget{benign-tuxfcmuxf6rler-1}{%
\section{Benign Tümörler}\label{benign-tuxfcmuxf6rler-1}}

\hypertarget{tuxfcbuxfcler-adenom-1}{%
\subsection{Tübüler Adenom}\label{tuxfcbuxfcler-adenom-1}}

\hypertarget{sesil-polip-flat-duxfcz-tuxfcbuxfcler-adenom-1}{%
\subsubsection{Sesil Polip, Flat (Düz) Tübüler
Adenom}\label{sesil-polip-flat-duxfcz-tuxfcbuxfcler-adenom-1}}

\url{https://pathologyatlas.github.io/tubularadenoma-flat/HE.html}

Mikroskopik görüntüleri inceleyin:

\url{https://pathologyatlas.github.io/tubularadenoma-flat/HE2.html}

Mikroskopik görüntüleri inceleyin:

\hypertarget{saplux131-polip-1}{%
\subsubsection{Saplı Polip}\label{saplux131-polip-1}}

Macroscopy

\href{https://pathologyatlas.github.io/tubularadenoma/tubular-adenoma-with-stalk-macroscopy.jpg}{tubular
adenoma with a stalk macroscopy}

\begin{figure}

{\centering \includegraphics{https://pathologyatlas.github.io/tubularadenoma/tubular-adenoma-with-stalk-macroscopy.jpg}

}

\caption{saplı polip}

\end{figure}

Microscopy

\href{https://pathologyatlas.github.io/tubularadenoma/tubular-adenoma-with-stalk.jpeg}{tubular
adenoma with a stalk}

Mikroskopik görüntüleri inceleyin:

\url{https://pathologyatlas.github.io/tubularadenoma/tubular-adenoma-with-stalk/viewer_z0.html}

\hypertarget{hiperplastik-polip}{%
\subsection{Hiperplastik Polip}\label{hiperplastik-polip}}

Mikroskopi

\url{https://pathologyatlas.github.io/hyperplasticpolyp/case1.html}

Mikroskopik görüntüleri inceleyin:

\hypertarget{hamartomatuxf6z-polip-1}{%
\subsection{Hamartomatöz Polip}\label{hamartomatuxf6z-polip-1}}

\url{https://pathologyatlas.github.io/hamartomatouspolyp/HE.html}

Mikroskopik görüntüleri inceleyin:

\hypertarget{schwann-cell-hamartoma-in-a-colon-polyp-1}{%
\subsubsection{Schwann Cell Hamartoma in a Colon
Polyp}\label{schwann-cell-hamartoma-in-a-colon-polyp-1}}

\url{https://pathologyatlas.github.io/schwanncellhamartoma/HE.html}

Mikroskopik görüntüleri inceleyin:

\hypertarget{kolon-intramukozal-lipom}{%
\chapter{Kolon intramukozal lipom}\label{kolon-intramukozal-lipom}}

\textbf{Kolon intramukozal lipom}

\url{https://pathologyatlas.github.io/colon-intramucosal-lipoma/HE.html}

Mikroskopik görüntüleri inceleyin:

\hypertarget{tuxfcbuxfcluxf6villuxf6z-adenom-zemininde-geliux15fmiux15f-muxfcsinuxf6z-adenokarsinom-kolon}{%
\chapter{Tübülövillöz adenom zemininde gelişmiş Müsinöz adenokarsinom,
kolon}\label{tuxfcbuxfcluxf6villuxf6z-adenom-zemininde-geliux15fmiux15f-muxfcsinuxf6z-adenokarsinom-kolon}}

\textbf{Tübülövillöz adenom zemininde gelişmiş Müsinöz adenokarsinom,
kolon}

\url{https://pathologyatlas.github.io/mucinous-adenocarcinoma-colon/HE.html}

Mikroskopik görüntüleri inceleyin:

\begin{center}\rule{0.5\linewidth}{0.5pt}\end{center}

\hypertarget{kolon-adenokarsinomu}{%
\chapter{Kolon Adenokarsinomu}\label{kolon-adenokarsinomu}}

\textbf{Kolon Adenokarsinomu}

\url{https://pathologyatlas.github.io/colon-adenocarcinoma/HE.html}

Mikroskopik görüntüleri inceleyin:

\part{Karaciğer Patolojisi}

\hypertarget{karaciux11fer-tuxfcmuxf6rleri}{%
\chapter{Karaciğer Tümörleri}\label{karaciux11fer-tuxfcmuxf6rleri}}

\hypertarget{hepatoseluxfcler-karsinom}{%
\section{Hepatoselüler Karsinom}\label{hepatoseluxfcler-karsinom}}

\url{https://pathologyatlas.github.io/hepatocellularcarcinoma/HCC/viewer_z0.html}

Mikroskopik görüntüleri inceleyin:

\hypertarget{hepatoseluxfcler-karsinom-fibrolamellar}{%
\section{Hepatoselüler karsinom,
fibrolamellar}\label{hepatoseluxfcler-karsinom-fibrolamellar}}

::: \{.cell\}

\textbf{Hepatoselüler karsinom, fibrolamellar}

\url{https://pathologyatlas.github.io/fibrolamellar-hepatocellular-carcinoma/HE1.html}

Mikroskopik görüntüleri inceleyin:

\url{https://pathologyatlas.github.io/fibrolamellar-hepatocellular-carcinoma/HE2.html}

Mikroskopik görüntüleri inceleyin:

\url{https://pathologyatlas.github.io/fibrolamellar-hepatocellular-carcinoma/HE3.html}

Mikroskopik görüntüleri inceleyin:

\url{https://pathologyatlas.github.io/fibrolamellar-hepatocellular-carcinoma/HE4.html}

Mikroskopik görüntüleri inceleyin:

:::

\part{Pankreatobilier Sistem Patolojisi}

\hypertarget{safra-kesesi-adenomyom}{%
\chapter{Safra Kesesi Adenomyom}\label{safra-kesesi-adenomyom}}

\textbf{Safra Kesesi Adenomyom}

\url{https://pathologyatlas.github.io/gallbladder-adenomyoma/HE.html}

Mikroskopik görüntüleri inceleyin:

\part{Mezotel}

\hypertarget{abdominal-mezotelyoma}{%
\chapter{Abdominal mezotelyoma}\label{abdominal-mezotelyoma}}

\textbf{Abdominal mezotelyoma}

\url{https://pathologyatlas.github.io/abdominal-mesothelioma/HE.html}

Mikroskopik görüntüleri inceleyin:

\part{Jinekolojik Patoloji}

\hypertarget{jinekopatoloji}{%
\chapter{Jinekopatoloji}\label{jinekopatoloji}}

\hypertarget{over}{%
\section{Over}\label{over}}

\hypertarget{ovary-serous-borderline-tumor-micropapillary-variant}{%
\subsection{Ovary, Serous borderline tumor, micropapillary
variant}\label{ovary-serous-borderline-tumor-micropapillary-variant}}

\url{https://pathologyatlas.github.io/ovarianserousmicropapillary/HE.html}

Mikroskopik görüntüleri inceleyin:

\hypertarget{genital-sistem-patolojisi}{%
\chapter{Genital Sistem Patolojisi}\label{genital-sistem-patolojisi}}

\hypertarget{endometrial-polip}{%
\chapter{Endometrial Polip}\label{endometrial-polip}}

\textbf{Endometrial Polip}

\url{https://pathologyatlas.github.io/endometrial-polyp/HE.html}

Mikroskopik görüntüleri inceleyin:

\part{Üriner Sistem Patolojisi}

\hypertarget{uriner-sistem-patolojisi}{%
\chapter{Uriner Sistem Patolojisi}\label{uriner-sistem-patolojisi}}

\hypertarget{buxf6brek-tuxfcmuxf6rleri}{%
\chapter{Böbrek Tümörleri}\label{buxf6brek-tuxfcmuxf6rleri}}

\hypertarget{onkositom}{%
\section{Onkositom}\label{onkositom}}

\url{https://pathologyatlas.github.io/kidneyoncocytoma/HE.html}

Mikroskopik görüntüleri inceleyin:

\part{Endokrin Sistem Patolojisi}

\hypertarget{hipofiz-adenomu}{%
\chapter{Hipofiz Adenomu}\label{hipofiz-adenomu}}

\textbf{Hipofiz Adenomu}

\url{https://pathologyatlas.github.io/pituitary-adenoma/HE.html}

Mikroskopik görüntüleri inceleyin:

\part{Meme Patolojisi}

\part{Breast Pathology}

See extended list of Breast Pathology cases here
\url{https://pathologyatlas.github.io/breast/}

\part{Baş Boyun Patolojisi}

\hypertarget{kulak}{%
\chapter{Kulak}\label{kulak}}

\hypertarget{kolesteatom}{%
\subsection{Kolesteatom}\label{kolesteatom}}

\url{https://pathologyatlas.github.io/cholesteatoma/cholesteatoma.html}

Mikroskopik görüntüleri inceleyin:

\part{Santral Sinir Sistem Patolojisi}

\hypertarget{yuxfcksek-dereceli-gliom-yayma-preparat}{%
\chapter{Yüksek Dereceli Gliom Yayma
Preparat}\label{yuxfcksek-dereceli-gliom-yayma-preparat}}

\textbf{Yüksek Dereceli Gliom Yayma Preparat}

\url{https://pathologyatlas.github.io/high-grade-glioma-squash/HE.html}

Mikroskopik görüntüleri inceleyin:

\hypertarget{beyin-invazyonu-guxf6steren-meningiom}{%
\chapter{Beyin İnvazyonu Gösteren
Meningiom}\label{beyin-invazyonu-guxf6steren-meningiom}}

\textbf{Beyin İnvazyonu Gösteren Meningiom}

\url{https://pathologyatlas.github.io/brain-invasive-meningioma/HE.html}

Mikroskopik görüntüleri inceleyin:

\hypertarget{meningiom-kemik-infiltrasyonu}{%
\chapter{Meningiom Kemik
İnfiltrasyonu}\label{meningiom-kemik-infiltrasyonu}}

\textbf{Meningiom Kemik İnfiltrasyonu}

\url{https://pathologyatlas.github.io/meningioma-bone-infiltration/HE.html}

Mikroskopik görüntüleri inceleyin:

\part{Kemik Patolojisi}

\hypertarget{kemik-tuxfcmuxf6rleri}{%
\chapter{Kemik Tümörleri}\label{kemik-tuxfcmuxf6rleri}}

\hypertarget{benign-kemik-tuxfcmuxf6rleri}{%
\section{Benign Kemik Tümörleri}\label{benign-kemik-tuxfcmuxf6rleri}}

\hypertarget{ekzostoz-osteokondrom}{%
\subsection{Ekzostoz (Osteokondrom)}\label{ekzostoz-osteokondrom}}

\url{https://pathologyatlas.github.io/exostosis/oc.html}

Mikroskopik görüntüleri inceleyin:

\url{https://pathologyatlas.github.io/exostosis/oc001.html}

Mikroskopik görüntüleri inceleyin:

\url{https://pathologyatlas.github.io/exostosis/oc002.html}

Mikroskopik görüntüleri inceleyin:

\hypertarget{section-1}{%
\subsection{}\label{section-1}}

\part{Yumuşak Doku Patolojisi}

\hypertarget{miksoid-liposarkom}{%
\chapter{Miksoid Liposarkom}\label{miksoid-liposarkom}}

\textbf{Miksoid Liposarkom}

\url{https://pathologyatlas.github.io/myxoidliposarcoma/HE.html}

Mikroskopik görüntüleri inceleyin:

\part{Pediatrik ve Perinatal Patoloji}

\hypertarget{plasenta}{%
\chapter{Plasenta}\label{plasenta}}

\hypertarget{koryoamnionit}{%
\section{Koryoamnionit}\label{koryoamnionit}}

\url{https://pathologyatlas.github.io/chorioamnionitis/HE.html}

Mikroskopik görüntüleri inceleyin:

\part{---}

\bookmarksetup{startatroot}

\hypertarget{referanslar}{%
\chapter*{Referanslar}\label{referanslar}}
\addcontentsline{toc}{chapter}{Referanslar}

\hypertarget{refs}{}
\begin{CSLReferences}{0}{0}
\end{CSLReferences}

\part{---}

\appendix
\addcontentsline{toc}{part}{Ekler}

\hypertarget{yuxf6netim-ve-geliux15ftirme}{%
\chapter{Yönetim ve Geliştirme}\label{yuxf6netim-ve-geliux15ftirme}}

\href{https://pathologyatlas.github.io/}{pathologyatlas.github.io}

\href{https://lab.patolojinotlari.com}{lab.patolojinotlari.com}

\href{https://patolojinotlari.com}{patolojinotlari.com}

\href{https://parapathology.com}{parapathology.com}

\href{https://twitter.com/patolojinotlari}{twitter}

\href{https://www.linkedin.com/company/patoloji-notlari}{linkedin}

\url{https://leanpub.com/patolojiatlasi/}

\href{https://pathologyatlas.github.io/development.md}{development \&
WIP}

\begin{itemize}
\item
  \url{https://github.com/pathologyatlas}
\item
  \url{https://github.com/pathologyatlas/TODO}
\item
  \url{https://github.com/pathologyatlas/pathologyatlas.github.io}
\item
  \url{https://github.com/pathologyatlas/template}
\item
  \url{https://github.com/pathologyatlas/make-html-WSI}
\end{itemize}

contact: info@patolojinotlari.com

contact: bilgi@patolojiatlasi.com

\begin{itemize}
\item
  https://github.com/marketplace/actions/quarto-render
\item
  https://github.com/quarto-dev/quarto-actions
\end{itemize}

\begin{center}\rule{0.5\linewidth}{0.5pt}\end{center}

\hypertarget{examples-for-future-uploads}{%
\chapter{Examples for future
uploads}\label{examples-for-future-uploads}}

\hypertarget{pancreas-ductal-adenocarcinoma}{%
\section{Pancreas Ductal
Adenocarcinoma}\label{pancreas-ductal-adenocarcinoma}}

\url{https://pathologyatlas.github.io/pancreaticadenocarcinoma/}

\hypertarget{case-1}{%
\subsection{Case 1}\label{case-1}}

\href{https://pathologyatlas.github.io/pancreaticadenocarcinoma/case1-histopathology/viewer_z0.html}{histopathology}

\hypertarget{neoplazinin-klinikopatolojik-uxf6zellikleri-ve-epidemiyoloji}{%
\section{Neoplazinin Klinikopatolojik Özellikleri ve
Epidemiyoloji}\label{neoplazinin-klinikopatolojik-uxf6zellikleri-ve-epidemiyoloji}}

\url{https://pathologyatlas.github.io/lecture1/Neoplazinin-Klinikopatolojik-Ozellikleri-ve-Epidemiyoloji.html}

\begin{center}\rule{0.5\linewidth}{0.5pt}\end{center}

\hypertarget{adding-submodule}{%
\chapter{adding submodule}\label{adding-submodule}}

https://github.blog/2016-02-01-working-with-submodules/

\begin{Shaded}
\begin{Highlighting}[]
\FunctionTok{git}\NormalTok{ submodule add https://github.com/pathologyatlas/pancreaticadenocarcinoma pancreaticadenocarcinoma}
\end{Highlighting}
\end{Shaded}

\hypertarget{removing-submodule}{%
\chapter{removing submodule}\label{removing-submodule}}

\begin{Shaded}
\begin{Highlighting}[]
\FunctionTok{git}\NormalTok{ submodule deinit }\AttributeTok{{-}f} \AttributeTok{{-}{-}all}
\FunctionTok{rm} \AttributeTok{{-}rf}\NormalTok{ .git/modules/}
\FunctionTok{git}\NormalTok{ rm }\AttributeTok{{-}f}\NormalTok{ pancreaticadenocarcinoma}
\end{Highlighting}
\end{Shaded}

\hypertarget{making-wsi}{%
\chapter{making WSI}\label{making-wsi}}

see:
\href{https://github.com/pathologyatlas/make-html-WSI\#convert-svs-to-dzi-and-publish-as-or-embed-in-a-web-page}{Convert
.svs to .dzi and publish as or embed in a web page}

\hypertarget{using-templates-for-new-repositories}{%
\chapter{using templates for new
repositories}\label{using-templates-for-new-repositories}}

see: \href{https://github.com/pathologyatlas/template}{Using templates
for new repositories}

\begin{center}\rule{0.5\linewidth}{0.5pt}\end{center}

\hypertarget{todo}{%
\chapter{TODO}\label{todo}}

\hypertarget{yuxf6nlendirme}{%
\section{Yönlendirme}\label{yuxf6nlendirme}}

\begin{itemize}
\tightlist
\item
  development ve wsi to html repolarının adreslerini güncelle ve
  detaylandır
\item
  https://quarto.org/docs/websites/website-navigation.html\#pages-404
\item
  Yönlendirme örneği: https://github.com/patolojiatlasi/GBD
\end{itemize}

\hypertarget{gbdatlas}{%
\section{GBDAtlas}\label{gbdatlas}}

\begin{itemize}
\tightlist
\item
  GBDAtlas ilk grubu ekle
\end{itemize}

\hypertarget{language}{%
\section{Language}\label{language}}

\begin{itemize}
\tightlist
\item
  https://statnmap.com/2017-04-04-format-text-conditionally-with-rmarkdown-chunks-complete-iframed-article/
\item
  https://statnmap.com/2017-10-06-translation-rmarkdown-documents-using-data-frame/
\item
  https://statnmap.com/2017-03-11-rmarkdown-conditional-chunks-to-create-multilingual-pdf-and-html-with-images/
\item
  https://bookdown.org/yihui/rmarkdown-cookbook/eng-asis.html
\item
  https://quarto.org/docs/authoring/create-citeable-articles.html
\item
  https://github.com/maelle/multilingual-book
\item
  https://github.com/patolojiatlasi/multilingual-book
\end{itemize}

\hypertarget{github-actions}{%
\section{GitHub Actions}\label{github-actions}}

\begin{itemize}
\tightlist
\item
  https://github.com/quarto-dev/quarto-actions
\item
  https://github.com/pommevilla/quarto-render
\item
  https://github.com/pommevilla/friendly-dollop
\item
  https://pommevilla.github.io/friendly-dollop/
\item
  https://github.com/dicook/github-action-push-to-another-repository
\item
  https://github.com/marketplace?page=1\&q=another+repo\&query=another+repo+\&type=actions
\item
  https://github.com/marketplace/actions/file-changes-action
\end{itemize}

\hypertarget{tweeting-with-update}{%
\section{tweeting with update}\label{tweeting-with-update}}

\begin{itemize}
\tightlist
\item
  https://github.com/and-computers/HowToTweetEveryCommit
\item
  https://www.daveabrock.com/2020/04/19/posting-to-twitter-from-gh-actions/
\item
  https://gitweet.io/
\item
  https://joshuaiz.com/words/using-github-actions-to-post-to-twitter-on-commit
\end{itemize}

\begin{center}\rule{0.5\linewidth}{0.5pt}\end{center}

\begin{Shaded}
\begin{Highlighting}[]
\FunctionTok{git}\NormalTok{ commit }\AttributeTok{{-}{-}allow{-}empty} \AttributeTok{{-}m} \StringTok{"Empty{-}Commit"}
\FunctionTok{git}\NormalTok{ push origin main}
\end{Highlighting}
\end{Shaded}

\begin{center}\rule{0.5\linewidth}{0.5pt}\end{center}

\begin{itemize}
\tightlist
\item
  https://www.w3schools.com/howto/howto\_css\_responsive\_iframes.asp
\end{itemize}

\begin{center}\rule{0.5\linewidth}{0.5pt}\end{center}

\begin{tcolorbox}[enhanced jigsaw, toptitle=1mm, arc=.35mm, titlerule=0mm, colframe=quarto-callout-note-color-frame, rightrule=.15mm, opacitybacktitle=0.6, colbacktitle=quarto-callout-note-color!10!white, toprule=.15mm, leftrule=.75mm, coltitle=black, breakable, bottomtitle=1mm, opacityback=0, title=\textcolor{quarto-callout-note-color}{\faInfo}\hspace{0.5em}{Not}, bottomrule=.15mm, left=2mm, colback=white]
Note that there are five types of callouts, including: \texttt{note},
\texttt{tip}, \texttt{warning}, \texttt{caution}, and
\texttt{important}.
\end{tcolorbox}

\begin{tcolorbox}[enhanced jigsaw, toptitle=1mm, arc=.35mm, titlerule=0mm, colframe=quarto-callout-tip-color-frame, rightrule=.15mm, opacitybacktitle=0.6, colbacktitle=quarto-callout-tip-color!10!white, toprule=.15mm, leftrule=.75mm, coltitle=black, breakable, bottomtitle=1mm, opacityback=0, title=\textcolor{quarto-callout-tip-color}{\faLightbulb}\hspace{0.5em}{Tip With Caption}, bottomrule=.15mm, left=2mm, colback=white]
This is an example of a callout with a caption.
\end{tcolorbox}

\begin{tcolorbox}[enhanced jigsaw, toptitle=1mm, arc=.35mm, titlerule=0mm, colframe=quarto-callout-caution-color-frame, rightrule=.15mm, opacitybacktitle=0.6, colbacktitle=quarto-callout-caution-color!10!white, toprule=.15mm, leftrule=.75mm, coltitle=black, breakable, bottomtitle=1mm, opacityback=0, title=\textcolor{quarto-callout-caution-color}{\faFire}\hspace{0.5em}{Expand To Learn About Collapse}, bottomrule=.15mm, left=2mm, colback=white]
This is an example of a `folded' caution callout that can be expanded by
the user. You can use \texttt{collapse="true"} to collapse it by default
or \texttt{collapse="false"} to make a collapsible callout that is
expanded by default.
\end{tcolorbox}

\hypertarget{katkux131da-bulunmak-iuxe7in}{%
\chapter{Katkıda Bulunmak İçin}\label{katkux131da-bulunmak-iuxe7in}}

H\&E ya da özel boyalı preparatlarınızı bize elden ya da kargo ile
ulaştırırsanız, onları tarayıp atlasa ekleyebiliriz.

Halihazırda taranmış preparatlarınızı bulut üzerinden
\href{mailto:bilgi@patolojiatlasi.com}{eposta} ile paylaşırsanız, onları
da ekleyebiliriz.

Preparatın olduğu sayfaya eklenmek üzere

\begin{verbatim}
Adınızı,

Kurumunuzu

Yönlendirilmesini istediğiniz iletişim linkini (web ya da e-posta)

Eklemek istediğiniz ek klinik bilgi, resim ya da notları
\end{verbatim}

\href{mailto:bilgi@patolojiatlasi.com}{\nolinkurl{bilgi@patolojiatlasi.com}}
adresine iletiniz.

Adres: Doç.Dr.Serdar Balcı Memorial Patoloji Laboratuvarı, Ortadoğu
Plaza Kat:14, Kaptanpaşa Mah. Piyalepaşa Bulvarı, Okmeydanı Cd No:73,
34384 Şişli/İstanbul

\hypertarget{section-4}{%
\chapter{}\label{section-4}}

\hypertarget{list}{%
\chapter{List}\label{list}}

\hypertarget{list-1}{%
\chapter{List}\label{list-1}}



\end{document}
